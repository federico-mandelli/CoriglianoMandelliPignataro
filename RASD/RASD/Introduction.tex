
% \ac{} Stampa l'acronimo completo solo la prima volta che lo si usa e in parentesi l'acronimo;
% \acf{} Stampa sempre l'acronimo completo e in parentesi l'acronimo;
% \acp{} Come \ac ma aggiunge la s alla fine per il plurale;
% \acl{} Stampa solo l'acronimo completo.

% Introduction

\subsection{Purpose}
Due to the recent increase of effort in the battle against climate change, electric vehicles are slowly becoming the new technology for private transport that   people use everyday.
To sustain this type of strategy, we need to develop a clever and capillary charging system.\\
\acf{eMall} is an \acf{eMSP} that aims to help the final users dealing with their charging needs by informing the users about the nearby charging stations, their cost, their environmental friendliness and any special offer that they might have.
It will allow the users to book, cancel and pay for a charge and it will notify them when the charging process is terminated.\\
With the integration of the user's calendar, the system will also suggest the best moment in the schedule to charge the vehicle.
To have a fully integrated system, all the \acfp{CPO} will have a technological support called \acf{CPMS} to interface the service with the physical charging stations and to manage all the energy sources like batteries and \acfp{DSO}.
\acp{CPMS} will be in charge of deciding the energy source and, in case of batteries in a charging station, they will also manage their charging.
\subsubsection{Goals}
%\input{Introduzione}
\begin{enumerate}[label=\textbf{G\arabic*}]
    \item The \ac{eMSP} shall help the user to select the station;\label{goal:eMSP-helps-selecting}
    \item The \ac{eMSP} shall allow the user to book, cancel and pay for a charge;\label{goal:eMSP-booking-charge}
    \item The \ac{eMSP} shall allow the user to perform a charge;\label{goal:eMSP-allow-charge}
    \item \acp{CPMS} shall handle the vehicle charging cycles;\label{goal:CPMS-handles-charge}
    \item \acp{CPMS} shall manage the charging stations;\label{goal:CPMS-manage-station}
\end{enumerate}

% Subsection where we enumerate all the world and shared phenomena
\subsection{Scope}

% world phenomena
\begin{enumerate}[label=\textbf{WP\arabic*}]
    \item People charge electric vehicles in different modes (NORMAL, FAST, SUPER-FAST); \label{world:people-charge-vehicles}
    \item People use web calendar; \label{world:people-use-calendars}
    \item People pay for their charging service; \label{world:people-pay-service}
    \item \acp{DSO} supply energy to \acp{CPO}; \label{world:DSO-supply-energy}
    \item Some \acp{CPO} own batteries; \label{world:CPO-own-batteries}
    \item \acp{CPO} have \ac{IBAN} and \gls{partita IVA};\label{world:CPO-have-partitaIVA}
    \item \acp{CPO} decide whether to use batteries or \ac{DSO} supplied energy; \label{world:CPO-decide-energy}
\end{enumerate}
% shared phenomena
\begin{enumerate}[label=\textbf{SP\arabic*}]
    % system -> user
    \item The \ac{eMSP} suggests the user to charge the vehicle; \label{shared:eMSP-suggests-charge}
    \item The \ac{eMSP} notifies the user when the charging process is finished; \label{shared:eMSP-notifies-charging-finished}
    \item \acp{CPMS} acquire information about energy prizes from \acp{DSO}; \label{shared:CPMS-info-from-DSO}
          % user -> system
    \item The user books a charge using the \ac{eMSP}; \label{shared:user-books-charge}
    \item The user asks the \ac{eMSP} for suggestions about charging station; \label{shared:user-asks-suggestions}
    \item The user pays for the service using the \ac{eMSP}; \label{shared:user-pays-service}
    \item \acp{CPO} gather the energy source through the \ac{CPMS}; \label{shared:CPO-energy-through-CPMS}
\end{enumerate}

\subsection{Definitions, Acronyms, Abbreviations}
\subsubsection{Definitions}

% Print the glossary 
\printnoidxglossaries


\subsubsection{Acronyms}
\begin{multicols}{2}[]
    \begin{acronym}[RASDacronyms]
        \acro{eMall}{e-Mobility for All}
        \acro{eMSP}{e-Mobility Service Provider}
        \acro{CPO}{Charging Point Operator}
        \acro{CPMS}{Charge Point Management System}
        \acro{DSO}{Distribution System Operator}
        \acro{API}{Application Programming Interface}
        \acro{RACS}{Reliable Array of Cloned Services}
        \acro{RAPS}{Reliable Array of Partitioned Services}
        \acro{GDPR}{General Data Protection Regulation}
        \acro{SoC}{State of Charge}
        \acro{GPS}{Global Positioning System}
        \acro{IBAN}{International Bank Account Number}
    \end{acronym}
\end{multicols}

\subsection{Revision history}
\begin{itemize}
    \item Added Use Case;
    \item Refined Scenarios;
    \item Refined Sequence diagrams;
    \item Refined UML class diagram;
    \item Minor details (typo) fixes;
\end{itemize}

\subsection{Document Structure}
The document is divided in six main sections:
\begin{itemize}
    \item \textbf{Introduction}: The introduction illustrates the problem to the reader and enumerates all the goals that
          the system needs to achieve. Formal descriptions about the world (world phenomena) and the interactions between the system and the world (shared phenomena) are provided. At the end of the introduction there is a reference subsection for definitions and revision history;
    \item \textbf{Overall Description}: It is an high level description of the dynamic interaction between stakeholders and the system. For this reason in this section there are
          the main scenarios descriptions and a UML diagram which specifies all the relations from an upper model perspective. There is a subsection that illustrates
          the fundamental requirements of the system and another which specifies the type and description of any user. At the end of this section there is a collection of assumptions that are made over the complete project;
    \item \textbf{Specific Requirements}: This section focuses on all the details introduced in the \textbf{Overall Description}, it formalizes all the requirements
          about the system and all the scenarios. For this reason, use cases and sequence diagrams are illustrated. More constraints on the performance, design aspects and attributes of the software are shown;
    \item \textbf{Formal Analysis with Alloy}: It represents a formal description of the problem in Alloy language, with some formal constraints that need to be satisfied (asserts).
          This formalization is useful to validate the model itself and to verify that all the assertions are granted.
    \item \textbf{Effort Spent}: Summarizes the total hours spent on the document formalization;
    \item \textbf{References}: Summarizes all the reference documents that we used during the description.
\end{itemize}
\newpage


% \def\goalsToPhenomena{
% {1,2}, 
% {3,4}}

% \foreach \g in \goalsToPhenomena{
%     \pgfmathparse{{\g}[0]}\pgfmathresult : \pgfmathparse{{\g}[1]}\pgfmathresult

% }
\clearpage